\documentclass{article}
\usepackage[utf8]{inputenc}
\renewcommand\thesection{\alph{section}}
\title{BIOS13 - Question 1}
\author{Pham Xuan Huy Nguyen}

\begin{document}

\maketitle

\section{Construction of a binary genetic algorithm}
1. Represent your problem as genes in a binary value
chromosome.\\
2. Random generation of a ”population” of chromosomes\\
3. Evaluate ”fitness” (how good is a chromosome as solution) of individual chromosomes\\
4. Sort solutions after evaluating.\\
5. Pairing, recombination and thus producing new offspring.\\
The process of pairing and recombination evolving elitistic pairing and random cut, for example: two highest ranking individual will undergo that process to produce high quality offspring. Next, the individual 3 and 4, and so on. \\
6. Insert offspring in population, discard worst solutions.\\
The previous offsprings that were made in 5 is inserted in the population, the low ranked individuals will be discarded to maintain the same number of chromosomes.\\
7. Mutation.\\
Introducing mutation to the whole population, except the individuals that stay at the top of the ranking (having high fitness).\\
8. Cycle until termination.\\
Step 3 to 7 will be repeated until having the best possible solution.

\section{Why does a single layer perceptron need a bias?}

Bias acts as an adjustable model parameter, which was added to a perceptron's weighted sum of inputs and weights, to make the model’s performance on training data, or fit the given data as good as possible, thus increasing the model accuracy.
\end{document}
